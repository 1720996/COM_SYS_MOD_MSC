\documentclass[12pt,letterpaper]{article}
\usepackage{fullpage}
\usepackage[top=2cm, bottom=4.5cm, left=2.5cm, right=2.5cm]{geometry}
\usepackage{amsmath,amsthm,amsfonts,amssymb,amscd}
\usepackage{lastpage}
\usepackage{enumerate}
\usepackage{fancyhdr}
\usepackage{mathrsfs}
\usepackage{xcolor}
\usepackage{graphicx}
\usepackage{listings}
\usepackage{hyperref}

\makeatletter
\def\old@comma{,}
\catcode`\,=13
\def,{%
  \ifmmode%
    \old@comma\discretionary{}{}{}%
  \else%
    \old@comma%
  \fi%
}
\makeatother


 


\setlength{\parindent}{0.0in}
\setlength{\parskip}{0.05in}

% Edit these as appropriate
\newcommand\course{Theory of Complex Networks}
\newcommand\hwnumber{1}                  % <-- homework number
\newcommand\NetIDa{Luke Dando}           % <-- NetID of person #1
\newcommand\NetIDb{21131620}           % <-- NetID of person #2 (Comment this line out for problem sets)

\pagestyle{fancyplain}
\headheight 35pt
\lhead{\NetIDa}
\lhead{\NetIDa\\\NetIDb}                 % <-- Comment this line out for problem sets (make sure you are person #1)
\chead{\textbf{\Large Homework \hwnumber}}
\rhead{\course \\ \today}
\lfoot{}
\cfoot{}
\rfoot{\small\thepage}
\headsep 1.5em

\begin{document}

\section*{Question 1}
\subsection*{Graph 1}


Simple \& directed.\\
$V = \{ 2,3,4,5,6,7,9,10,11 \}$.\\
$E = \{ (2,3),(3,4),(3,5),(3,7),(3,9),(4,7),(6,3),(10,3),(11,3) \}$.
\subsection*{Graph 2}


Simple \& non-directed.\\
$V = \{ 1,2,3,4,5,6,7,8,9 \}$.\\
$E = \{ (1,2),(2,1),(2,3),(2,4),(2,5),(2,6),(2,7),(2,8),(2,9),(3,2),(4,2),(5,2),(6,2),(7,2),(8,2),(9,2) \}$.


\subsection*{Graph 3}
Simple \& non-directed.\\
$V = \{ 1,2,3,4,5 \}$.\\
$E = \{ (1,2),(1,4),(2,1),(2,5),(3,4),(4,1),(4,3),(4,5),(5,2),(5,4) \}$.



\section*{Question 2 (submitted for feedback)}
\subsection*{Graph 1}
Relabel $V=\{ 2,3,4,5,6,7,9,10,11 \} \to \{ 1,2,3,4,5,6,7,8,9 \}$.
\begin{equation}
    \mathbf{A}_1 =  \begin{pmatrix}
        0 & 0 & 0 & 0 & 0 & 0 & 0 & 0 & 0 \\
        1 & 0 & 0 & 0 & 1 & 0 & 0 & 1 & 1 \\
        0 & 1 & 0 & 0 & 0 & 0 & 0 & 0 & 0 \\
        0 & 1 & 0 & 0 & 0 & 0 & 0 & 0 & 0 \\
        0 & 0 & 0 & 0 & 0 & 0 & 0 & 0 & 0 \\
        0 & 1 & 1 & 0 & 0 & 0 & 0 & 0 & 0 \\
        0 & 1 & 0 & 0 & 0 & 0 & 0 & 0 & 0 \\
        0 & 0 & 0 & 0 & 0 & 0 & 0 & 0 & 0 \\
        0 & 0 & 0 & 0 & 0 & 0 & 0 & 0 & 0 
    \end{pmatrix}.
\end{equation}


\subsection*{Graph 2}
\begin{equation}
    \mathbf{A}_2 =  \begin{pmatrix}
        0 & 1 & 0 & 0 & 0 & 0 & 0 & 0 & 0 \\
        1 & 0 & 1 & 1 & 1 & 1 & 1 & 1 & 1 \\
        0 & 1 & 0 & 0 & 0 & 0 & 0 & 0 & 0 \\
        0 & 1 & 0 & 0 & 0 & 0 & 0 & 0 & 0 \\
        0 & 1 & 0 & 0 & 0 & 0 & 0 & 0 & 0 \\
        0 & 1 & 0 & 0 & 0 & 0 & 0 & 0 & 0 \\
        0 & 1 & 0 & 0 & 0 & 0 & 0 & 0 & 0 \\
        0 & 1 & 0 & 0 & 0 & 0 & 0 & 0 & 0 \\
        0 & 1 & 0 & 0 & 0 & 0 & 0 & 0 & 0 
    \end{pmatrix}  .
\end{equation}


\subsection*{Graph 3}
\begin{equation}
    \mathbf{A_3} = \begin{pmatrix}
        0 & 1 & 0 & 1 & 0 \\
        1 & 0 & 0 & 0 & 1 \\
        0 & 0 & 0 & 1 & 0 \\
        1 & 0 & 1 & 0 & 1 \\
        0 & 1 & 0 & 1 & 0 
    \end{pmatrix}.
\end{equation}



\section*{Question 3}
Let $\mathbf{u}_0$ be a $(1\times9)$-vector of $1$s. We then use our adjacency matrix $\mathbf{A}_1\equiv\mathbf{A}$ to help calculate the total number of paths of a given length $n$. This is expressed as follows
\begin{equation}
    p_{n} = \vert \mathbf{u}_n\vert_1 = \vert \mathbf{u}_0(\mathbf{A})^{n}\vert_1,\quad n\geq 1.
\end{equation}
Here, $\vert \mathbf{u}\vert_1$ is the $1$-norm of the vector $u$, summing the modulus of it's elements. This gives u the number of paths of length $n$ available, evaluated as $p_n$. We can then work through each value of $n$ up to $4$ (although it isn't necessary as it isn't an iterative process).
\begin{align}
    \mathbf{u}_1 &= \begin{pmatrix}
        1 & 1 & 1 & 1 & 1 & 1 & 1 & 1 & 1 
    \end{pmatrix}
    \begin{pmatrix}
        0 & 0 & 0 & 0 & 0 & 0 & 0 & 0 & 0 \\
        1 & 0 & 0 & 0 & 1 & 0 & 0 & 1 & 1 \\
        0 & 1 & 0 & 0 & 0 & 0 & 0 & 0 & 0 \\
        0 & 1 & 0 & 0 & 0 & 0 & 0 & 0 & 0 \\
        0 & 0 & 0 & 0 & 0 & 0 & 0 & 0 & 0 \\
        0 & 1 & 1 & 0 & 0 & 0 & 0 & 0 & 0 \\
        0 & 1 & 0 & 0 & 0 & 0 & 0 & 0 & 0 \\
        0 & 0 & 0 & 0 & 0 & 0 & 0 & 0 & 0 \\
        0 & 0 & 0 & 0 & 0 & 0 & 0 & 0 & 0 
    \end{pmatrix}, \\
    &= \begin{pmatrix}
        1 & 4 & 1 & 0 & 1 & 0 & 0 & 1 & 1 
    \end{pmatrix},\\
    &\to \vert\mathbf{u}_1\vert_1 = p_1 = 9.
\end{align}
This indicates there are a total of $9$ different ways of traversing a path of length $1$ exactly. This can be drawn trivially by hand to consolidate. \\
Next, $n=2$;
\begin{align}
    \mathbf{u}_2 &= \begin{pmatrix}
        1 & 4 & 1 & 0 & 1 & 0 & 0 & 1 & 1 
    \end{pmatrix}
    \begin{pmatrix}
        0 & 0 & 0 & 0 & 0 & 0 & 0 & 0 & 0 \\
        1 & 0 & 0 & 0 & 1 & 0 & 0 & 1 & 1 \\
        0 & 1 & 0 & 0 & 0 & 0 & 0 & 0 & 0 \\
        0 & 1 & 0 & 0 & 0 & 0 & 0 & 0 & 0 \\
        0 & 0 & 0 & 0 & 0 & 0 & 0 & 0 & 0 \\
        0 & 1 & 1 & 0 & 0 & 0 & 0 & 0 & 0 \\
        0 & 1 & 0 & 0 & 0 & 0 & 0 & 0 & 0 \\
        0 & 0 & 0 & 0 & 0 & 0 & 0 & 0 & 0 \\
        0 & 0 & 0 & 0 & 0 & 0 & 0 & 0 & 0 
    \end{pmatrix}, \\
    &= \begin{pmatrix}
        4 & 1 & 0 & 0 & 4 & 0 & 0 & 4 & 4 
    \end{pmatrix},\\
    &\to \vert\mathbf{u}_2\vert_1 = p_2 = 17.
\end{align}
This indicates there are a total of $17$ different ways of traversing a path of length $2$ exactly. \\
Next, $n=3$;
\begin{align}
    \mathbf{u}_3 &= \begin{pmatrix}
        4 & 1 & 0 & 0 & 4 & 0 & 0 & 4 & 4 
    \end{pmatrix}
    \begin{pmatrix}
        0 & 0 & 0 & 0 & 0 & 0 & 0 & 0 & 0 \\
        1 & 0 & 0 & 0 & 1 & 0 & 0 & 1 & 1 \\
        0 & 1 & 0 & 0 & 0 & 0 & 0 & 0 & 0 \\
        0 & 1 & 0 & 0 & 0 & 0 & 0 & 0 & 0 \\
        0 & 0 & 0 & 0 & 0 & 0 & 0 & 0 & 0 \\
        0 & 1 & 1 & 0 & 0 & 0 & 0 & 0 & 0 \\
        0 & 1 & 0 & 0 & 0 & 0 & 0 & 0 & 0 \\
        0 & 0 & 0 & 0 & 0 & 0 & 0 & 0 & 0 \\
        0 & 0 & 0 & 0 & 0 & 0 & 0 & 0 & 0 
    \end{pmatrix}, \\
    &= \begin{pmatrix}
        1 & 0 & 0 & 0 & 1 & 0 & 0 & 1 & 1 
    \end{pmatrix},\\
    &\to \vert\mathbf{u}_3\vert_1 = p_3 =  4.
\end{align}
This indicates there are a total of $17$ different ways of traversing a path of length $2$ exactly. This is also the solution to the first part of this question. More specifically, the $4$ paths of length three are;
\begin{equation}
    (1,5,8,9)\to(2)\to(3)\to(6).
\end{equation}
Finally, we need to show there are $0$ paths of length $n=4$.
\begin{align}
    \mathbf{u}_4 &= \begin{pmatrix}
        1 & 0 & 0 & 0 & 1 & 0 & 0 & 1 & 1 
    \end{pmatrix}
    \begin{pmatrix}
        0 & 0 & 0 & 0 & 0 & 0 & 0 & 0 & 0 \\
        1 & 0 & 0 & 0 & 1 & 0 & 0 & 1 & 1 \\
        0 & 1 & 0 & 0 & 0 & 0 & 0 & 0 & 0 \\
        0 & 1 & 0 & 0 & 0 & 0 & 0 & 0 & 0 \\
        0 & 0 & 0 & 0 & 0 & 0 & 0 & 0 & 0 \\
        0 & 1 & 1 & 0 & 0 & 0 & 0 & 0 & 0 \\
        0 & 1 & 0 & 0 & 0 & 0 & 0 & 0 & 0 \\
        0 & 0 & 0 & 0 & 0 & 0 & 0 & 0 & 0 \\
        0 & 0 & 0 & 0 & 0 & 0 & 0 & 0 & 0 
    \end{pmatrix}, \\
    &= \begin{pmatrix}
        0 & 0 & 0 & 0 & 0 & 0 & 0 & 0 & 0 
    \end{pmatrix} = \mathbf{0},\\
    &\to \vert\mathbf{u}_4\vert_1 = p_4 = 0.
\end{align}
We can see that there are no paths for length $n=4$. Moreover, as we now have the zero-vector, all subsequent path lengths are forever $0$ in number.



\section*{Question 4}
\subsection*{Graph 1}
\begin{align}
    i &= (1,2,3,4,5,6,7,8,9), \\ 
    k_i^{in}(\mathbf{A}_1) &= (0,4,1,1,0,2,1,0,0), \\
    k_i^{out}(\mathbf{A}_1) &= (1,4,1,0,1,0,0,1,1).
\end{align}


\subsection*{Graph 2}
\begin{align}
    i &= (1,2,3,4,5,6,7,8,9), \\ 
    k_i(\mathbf{A}_2) &= (1,8,1,1,1,1,1,1,1).
\end{align}


\subsection*{Graph 3}
\begin{align}
    i &= (1,2,3,4,5), \\ 
    k_i(\mathbf{A}_3) &= (2,2,1,3,2).
\end{align}



\section*{Question 5}
For a non-directed graph, as for any edge $(i,j)$, there exists the reverse edge $(j,i)$ such that
\begin{equation}
    A_{ij}=A_{ji}\in\{1,0\}. \label{eq:non-directed}
\end{equation} Therefore
\begin{equation}
    k_i^{in}(\mathbf{A}) = \sum_{n=0}^N A_{ij} = \sum_{n=0}^N A_{ji} = k_1^{out}(\mathbf{A}) = k_i(\mathbf{A}).
\end{equation}
Let a non-directed $(n\times n)$-graph $\mathbf{G}$ be represented as
\begin{equation}
    \mathbf{G} = \begin{pmatrix}
        G_{11} & \cdots & G_{1n} \\
        \vdots & \ddots & \vdots \\
        G_{n1} & \cdots & G_{nn}  
    \end{pmatrix}.
\end{equation}
We then want to look at the diagonal of $\mathbf{G}^2$, which is represented as
\begin{equation}
    \mathbf{G}^2 = \begin{pmatrix}
        \sum_{k=1}^nG_{1k}G_{k1} &  &  &  \\
        & \sum_{k=1}^nG_{2k}G_{k2} &  &  \\
         &  & \ddots &  \\
         &  &  & \sum_{k=1}^nG_{nk}G_{kn}
    \end{pmatrix}.
\end{equation}
Therefore, we can conclude
\begin{equation}
    (\mathbf{G}^2)_{ii} = \sum_{k=1}^nG_{ik}G_{ki}.
\end{equation}
As it is a non-directed graph, we us eq.~(\ref{eq:non-directed}) to therefore conclude
\begin{equation}
    (\mathbf{G}^2)_{ii} = \sum_{k=1}^nG_{ik} = \sum_{k=1}^nG_{ki} = k_i(\mathbf{G}).
\end{equation}


\section*{Question 6}
We know that to evaluate the in-degree of a vertex $i$ in $\mathbf{A}$, we find
\begin{equation}
    k_i^{in}(\mathbf{A}) = \sum_{n=1}^N A_{i,n}.
\end{equation}
Therefore, for the sum of all in-degrees we have
\begin{equation}
    \sum_{i=1}^N k_i^{in}(\mathbf{A}) = \sum_{i,n=1}^N A_{i,n}.
\end{equation}
For the out-degree we then have
\begin{equation}
    \sum_{i=1}^N k_i^{out}(\mathbf{A}) = \sum_{i,n=1}^N A_{n,i}.
\end{equation}
As both $i$ and $n$ span the same set of values from set $V$;
\begin{equation}
    \sum_{i,n=1}^N A_{i,n} = \underbrace{\sum_{i=1}^N k_i^{in}(\mathbf{A}) = \sum_{i=1}^N k_i^{out}(\mathbf{A})} = \sum_{i,n=1}^N A_{n,i}.
\end{equation}

\end{document}
